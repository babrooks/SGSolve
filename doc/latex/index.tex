\hypertarget{index_introsec}{}\section{Introduction}\label{index_introsec}
S\+G\+Solve is a software package for representing, solving, and analyzing the equilibria of infinitely repeated stochastic games with two players, perfect monitoring, and a public randomization device. The library implements algorithms for computing the subgame perfect equilibrium payoff correspondence that was invented by Dilip Abreu, Ben Brooks, and Yuliy Sannikov (Abreu, Brooks, and Sannikov, 2016, 2019).

The package consists of two main components. The first piece is a library of C++ classes for representing and solving stochastic games, and the second is a graphical user interface (S\+G\+Viewer) for using and interacting with the solver library. The purpose of this guide is to give advanced users an overview of how the library and graphical interface are structured.

S\+G\+Solve makes use of external software packages\+: the Boost libraries are used for serialalization of data relating to stochastic games, which are represented with the \hyperlink{classSGGame}{S\+G\+Game} class and are saved as .sgm files, and the solutions of stochastic games that are generated by the algorithmm, which are represented with the \hyperlink{classSGSolution}{S\+G\+Solution} and \hyperlink{classSGIteration}{S\+G\+Iteration} classes and are saved as .sln files. The graphical interface uses the Qt framework and the Qt plotting library Q\+Custom\+Plot. Gurobi is used for various extensions using linear programming.\hypertarget{index_installsec}{}\section{Installation}\label{index_installsec}
To use this software, you can directly download the S\+G\+Viewer binaries, which are precompiled for Linux and OS X. The source code is also available at www.\+benjaminbrooks.\+net/research.shtml. The code has most recently been compiled on OS X using L\+L\+VM version 10.\+0.\+0 and Boost 1.\+69. The S\+G\+Viewer was compiled using Qt 5.\+4. The class S\+G\+J\+Y\+C\+Solver uses Gurobi for linear programming, and the code was compiled with Gurobi 8.\+00. To compile the code yourself, you need to change the relevant variables in localsettings.\+mk in the root directory. You can then build the desired examples by calling \char`\"{}make\char`\"{} in either the src or examples directories. After you build the source, you can build S\+G\+Viewer program from the viewer directory by first calling \char`\"{}qmake\char`\"{} and then \char`\"{}make\char`\"{}.\hypertarget{index_solversec}{}\section{Overview of the solver library}\label{index_solversec}
The machinery underlying the S\+G\+Solve package is a library of routines for specifying and solving stochastic games. For examples of how to use the library, see risksharing.\+cpp and risksharing.\+hpp. The main classes are \hyperlink{classSGGame}{S\+G\+Game}, which is used to specify a game, and the solver class \hyperlink{classSGSolver__MaxMinMax}{S\+G\+Solver\+\_\+\+Max\+Min\+Max}, whose routine \hyperlink{classSGSolver__MaxMinMax_aad121e84c1492524e439ffba05893f3d}{S\+G\+Solver\+\_\+\+Max\+Min\+Max\+::solve()} implements the max-\/min-\/max algorithm of Abreu, Brooks, and Sannikov (2019). The \hyperlink{classSGSolver}{S\+G\+Solver} class implements the deprecated pencil-\/sharpening algorithm of Abreu, Brooks, and Sannikov (2016).

An \hyperlink{classSGGame}{S\+G\+Game} can be constructed in one of two ways. First, the user can specify payoffs and transition probabilities as arrays and pass them to the \hyperlink{classSGGame}{S\+G\+Game} constructor. Alternatively, the user can create a class that derives from \hyperlink{classSGAbstractGame}{S\+G\+Abstract\+Game}, and pass an object of the derived class to the constructor for \hyperlink{classSGGame}{S\+G\+Game}. \hyperlink{classSGAbstractGame}{S\+G\+Abstract\+Game} contains virtual methods for retrieving the payoffs and transition probabilities that can be defined by the user. For an example of how to derive from \hyperlink{classSGAbstractGame}{S\+G\+Abstract\+Game}, see risksharing.\+hpp. Once the game is constructed, the user can construct an \hyperlink{classSGSolver}{S\+G\+Solver} for that game. \hyperlink{classSGSolver__MaxMinMax}{S\+G\+Solver\+\_\+\+Max\+Min\+Max}.\hypertarget{index_viewersec}{}\section{Overview of the graphical interface}\label{index_viewersec}
In addition to using the object model directly, the user can also interact with the S\+G\+Solve library through the S\+G\+Viewer graphical interface. This interface was constructed using the Qt framework and the Q\+Custom\+Plot plotting library (\href{http://www.qcustomplot.com/}{\tt http\+://www.\+qcustomplot.\+com/}).

The interface consists of three tabs. The \char`\"{}game tab\char`\"{} is for viewing and specifying a stochastic game. It contains tables that display, for one state at a time, the players\textquotesingle{} payoffs and the transition probabilities for each pair of actions. The user can edit payoffs, probabilities, and the discount factor, as well as add and delete actions and states.

From the game tab, the user can invoke the solve routine. The progress of the algorithm is displayed on the \char`\"{}log tab\char`\"{}.

Once the algorithm finishes, the output is displayed in a \char`\"{}solution
tab\char`\"{}. There are two tabs, one for the deprecated pencil-\/sharpening algorithm, and the other for the more powerful max-\/min-\/max algorithm. On the right-\/hand side of these tabs are a series of plots that display payoffs state-\/by-\/state. On the left-\/hand side are more detailed plots that decompose how payoffs are generated in one of the states.

Right-\/clicking on a plot brings up additional options for the user. The first option, \char`\"{}inspect point\char`\"{}, will show how a given payoff is decomposed into flow payoffs and continuation utilities. The second option, \char`\"{}simulate\char`\"{}, brings up another window for forward simulating the equilibrium that generates the given payoff.\hypertarget{index_examplesec}{}\section{Examples}\label{index_examplesec}
For the benefit of the user, we have included several examples of how to use the S\+G\+Solve package. The file pd\+\_\+twostate.\+cpp is an example of a two-\/state game, where the stage game in each state takes the form of a prisoner\textquotesingle{}s dilemma. This file shows how to construct an \hyperlink{classSGGame}{S\+G\+Game} by specifying the payoffs and transition probabilities as arrays.

The second example is risksharing.\+cpp, which constructs a risk sharing game a la Kocherlakota (1996), in which the two players have stochastic endowments and concave utility, and can insure one another against income shocks with transfers. The file risksharing.\+hpp constructs a risk sharing game by deriving from the \hyperlink{classSGAbstractGame}{S\+G\+Abstract\+Game} class, and risksharing.\+cpp uses that class to solve for a variety of parameter values.

These examples are reported in Abreu, Brooks, and Sannikov (2019).\hypertarget{index_conclusionsec}{}\section{Final thoughts}\label{index_conclusionsec}
The package has many more features that the user will hopefully discover. Within the src folder is a src/\+M\+A\+T\+L\+AB subfolder, that contains tools for interfacing between M\+A\+T\+L\+AB and S\+G\+Solve. In particular, sgmex2.\+cpp is a mex program that can be used to specify, solve, and analyze games from within M\+A\+T\+L\+AB. There is also a matlab m-\/file sgmexbuild2.\+m that was used to build sgmex on OS X with M\+A\+T\+L\+AB R2018a.

This program would not have been possible without the support of numerous groups and the contributions of others, and it is entirely fitting and appropriate that their contributions should be acknowledged. In particular, this program was developed with and incorporates elements of a number of other open source projects, including the Qt application framework (www.\+qt.\+io), Emmanuel Eichhammer\textquotesingle{}s Q\+Custom\+Plot (www.\+qcustomplot.\+com), the Boost libraries (www.\+boost.\+org), and the G\+NU project (www.\+gnu.\+org). I would also like to gratefully acknowledge support from the Becker Friedman Institute, the University of Chicago, and the National Science Foundation.

Finally, it should go without saying that this program is a work in progress. Feedback, bug reports, and contributions are much appreciated.

Enjoy!

Ben Brooks Chicago, IL \href{mailto:ben@benjaminbrooks.net}{\tt ben@benjaminbrooks.\+net} 